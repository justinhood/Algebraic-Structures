\documentclass[letterpaper,10pt]{article}
\usepackage[top=2cm, bottom=1.5cm, left=1cm, right=1cm]{geometry}
\usepackage{amsmath, amssymb, amsthm,graphicx, enumitem}
\usepackage{fancyhdr}
\pagestyle{fancy}

\lhead{\today}
\chead{MATH 710 Assignment 3}
\rhead{Justin Hood}

\newcommand{\Z}{\mathbb{Z}}
\newcommand{\Q}{\mathbb{Q}}
\newcommand{\R}{\mathbb{R}}
\newcommand{\C}{\mathbb{C}}
\newcommand{\inner}[2]{\langle #1,#2\rangle}
\newtheorem{lem}{Lemma}

\begin{document}
\begin{description}
\item[] \textbf{Question 1.}
\begin{enumerate}[label=\alph*.]
\item State and prove the Schwarz inequality.\\
For any non-zero vectors $u$ and $v$ with innerproduct $\inner{}{}$
\[|\inner{x}{y}|\leq||u||\ ||v||\]
\begin{proof}
Let $u$ and $v$ be non-zero vectors with inner product, $\inner{}{}$, and let $x$ be a scalar. Consider now,
\[\inner{u+xv}{u+xv}=\inner{u}{u}+x\inner{u}{v}+x\inner{u}{v}+x^2\inner{v}{v}=\inner{u}{u}+2x\inner{u}{v}+x^2\inner{v}{v}\]
By definition of the inner product, we know that it is greater than zero, so we have,
\[\inner{u}{u}+2x\inner{u}{v}+x^2\inner{v}{v}\geq 0\]
We shall expand this into the familiar quadratic form,
\[\underbrace{||v||}_{\text{a}}x^2+\underbrace{2\inner{u}{v}}_{\text{b}}x+\underbrace{||u||^2}_{\text{c}}\geq 0\]
For a quadratic function to be greater than or equal to zero, we need the $D\leq 0$, where $D$ is the discriminant of the quadratic formula. Because this condition is satisfied, we see that,
\begin{align*}
4\inner{u}{v}^2-4||u||^2||v||^2 &\leq 0\\
\Leftrightarrow \inner{u}{v}^2 &\leq ||u||^2||v||^2
\end{align*}
Taking the principle root of both sides, we arrive at the equation,
\[|\inner{u}{v}|\leq ||u||||v||\]
Thus, we have proven the Schwarz inequality
\end{proof}
\item Prove for any $p$ and $q$ in $\R^n$,
\[||p-q||\geq ||p||-||q||\]
where $||||$ is the magnitude of a vector in $\R^n$ defined by $||x||=\bigg(\sum_{i}x_i\bigg)^{1/2}$ and the dot product of two vectors $x,y$ is defined as $x\cdot y=\sum_ix_iy_i$, which is the inner product in $\R^n$.
\begin{proof}
We consider,
\[\big(||p||-||q||\big)^2\]
Expanding, we arrive at,
\[||p||^2-2||p||\ ||q||+||q||^2\]
Note, from the Schwarz inequality,
\[|\inner{x}{y}|\leq||u||\ ||v|| \Leftrightarrow -2|\inner{x}{y}|\geq -2||u||\ ||v||\]
Thus,
\[||p||^2-2||p||\ ||q||+||q||^2 \leq ||p||^2-2\inner{p}{q}+||q||^2\]
But,
\[||p-q||^2 = ||p||^2-2 p\cdot q +||q||^2=||p||^2-2\inner{p}{q}+||q||^2\]
So, we have
\[||p-q||^2\geq\big(||p||-||q||\big)^2\]
Taking the principle root, we arrive at,
\[||p-q||\geq\big|||p||-||q||\big|\]
Finally, we note that $x\leq|x|\ \forall x\in \R$. So,
\[||p-q||\geq\big|||p||-||q||\big|\geq ||p||-||q||\]
As desired.
\end{proof}
\item Now, let $\theta$ be the angle between vector $p$ and $q$. We define,
\[\cos(\theta)=\frac{p\cdot q}{||p||\ ||q||}\]
Consider now, the Schwarz inequality as applied to $\R^n$. Our inner product becomes the dot product, and we have,
\begin{align*}
p\cdot q &\leq||p||\ ||q||\\
-||p||\ ||q||\leq  p\cdot q &\leq||p||\ ||q||\\
-1\leq \frac{p\cdot q}{||p||\ ||q||}&\leq1
\end{align*}
So, we see that the range of $\cos(\theta)$ is $[-1,1]$
\end{enumerate}
\item \textbf{Question 2.}
\begin{enumerate}[label=\alph*.]
\item A subset $U$ of a vector space $V$ is a subspace of $V$ if $U$ is a vector space with the same operations as $V$.
\item Prove that the intersection between two subspaces of a vector space is also a subspace.
\begin{proof}
Let $A$ and $B$ be distinct subspaces of the vector space $V$, and let $W=A\cap B$.\\
First, we note that by the definition of a subspace, $\vec{0}\in A,B$. Thus, $\vec{0}\in W$. So, we have the zero vector.\\
Second, consider arbitrary elements $x,y\in W$. By construction of $W$, we know that $x,y\in A$ and $x,y\in B$. Because $A$ and $B$ are subspaces of $V$, we know that $x+y\in A$ and $x+y\in B$. Thus, we see that $x+y\in W$. So, $W$ is closed under addition.\\
Finally, we consider $c$, a scalar value. Then, $\forall x\in W$, we know that both $x$ and $cx$ are in $A$ and $B$ because they are subspaces. Thus, $cx\in W\ \forall x\in W$, and we see that $W$ is closed under scalar multiplication.\\
So, we may conclude that $W$ is also a subspace of $V$.
\end{proof}
\item Prove that the union between two subspaces of a vector space is also a subspace.\\
\begin{proof}
We consider an example. Let $A=\{(x,0)|x\in \R\}$, and $B=\{(0,y)|y\in \R\}$, i.e. the $x$ and $y$ axes. We know that these sets are subspaces of the vector space $\R^2$. Let us now construct,
\[W=\{(x,0),(0,y)|x,y\in \R\}=A\cup B\]
and consider,
\[x^*=(1,0),\ y^*(0,1)\]
By definition, $x^*$ and $y^*$ are in $W$, but,
\[x^*+y^*=(1,1)\not\in W\]
So, $W$ is not closed under addition, a violation of the definition of a vector subspace. Thus, we have reached a contradiction. Hence, in general, a set formed from the union of two vector subspaces will not also be a subspace.
\end{proof}
\end{enumerate}
\item \textbf{Question 3.} Let,
\[A=\begin{bmatrix}
2 & 1 \\
3 & -1
\end{bmatrix}\]
\begin{enumerate}[label=\alph*.]
\item Compute $A^TA^{-1}$. First, we compute $A^{-1}$ by finding $det(A)$.\[\begin{vmatrix}
2 & 1 \\
3 & -1
\end{vmatrix}=(2)(-1)-(1)(3)=-5\]
Then,
\[A^{-1}=\frac{1}{-5}\begin{bmatrix}
-1 & -1 \\
-3 & 2
\end{bmatrix}\]
So, we may compute,
\[A^TA^{-1}=\frac{1}{-5}\begin{bmatrix}
2 & 3\\
1 & -1
\end{bmatrix}\begin{bmatrix}
-1 & -1\\
-3 & 2
\end{bmatrix}=\frac{1}{-5}\begin{bmatrix}
-2-9 & -2+6 \\
-1+3 & -1-2
\end{bmatrix}=\begin{bmatrix}
\frac{11}{5} & \frac{-4}{5}\\
\frac{-2}{5} & \frac{3}{5}
\end{bmatrix}\]
\item A matrix $A$ is called symmetric if
\[A=A^T\]
Here,
\[A=\begin{bmatrix}
2 & 1 \\
3 & -1
\end{bmatrix}\neq \begin{bmatrix}
2 & 3 \\
1 & -1
\end{bmatrix}=A^T\]
So $A$ is not symmetric.
\item Let $f(x)=2x^2-2$. What is the trace of $f(A)$?\\
We begin by computing
\[f(A)=2A^2-2I\]
\[A^2=\begin{bmatrix}
2 & 1 \\
3 & -1
\end{bmatrix} \begin{bmatrix}
2 & 1 \\
3 & -1
\end{bmatrix}=\begin{bmatrix}
4+3 & 2-1 \\
6-3 & 3+1
\end{bmatrix}=\begin{bmatrix}
7 & 1\\
3 & 4
\end{bmatrix} \]
Then,
\[f(A)=2A^2-2I=2\begin{bmatrix}
7 & 1\\
3 & 4
\end{bmatrix}-2\begin{bmatrix}
1 & 0\\
0 & 1
\end{bmatrix}=\begin{bmatrix}
14 & 2\\
6 & 8
\end{bmatrix}-\begin{bmatrix}
2 & 0\\
0 & 2
\end{bmatrix}=\begin{bmatrix}
12 & 2\\
6 & 6
\end{bmatrix} \]
Then,
\[tr(f(A))=\sum_iA_{ii}=12+6=18\]
\end{enumerate}
\item \textbf{Question 4.}
Let A be any 2x2 matrix. Then,
\[\frac{1}{2}(tr(A))^2-\frac{1}{2}(tr(A^2))-det(A)=0\]
\begin{proof}
Let $A$ be an arbirtrary matrix of dimension $2\times2$. With elements $a,b,c,d$ constructed as,
\[A=\begin{bmatrix}
a & b\\
c & d
\end{bmatrix}\]
Consider first,
\[tr(A)=a+d\]
Then
\[\frac{1}{2}(tr(A))^2=\frac{1}{2}(a+d)^2=\frac{a^2}{2}+ad+\frac{d^2}{2}\]
Next,
\[A^2=AA=\begin{bmatrix}
a & b\\
c & d
\end{bmatrix}\begin{bmatrix}
a & b\\
c & d
\end{bmatrix}=\begin{bmatrix}
a^2+bc & ab+bd\\
ac+cd & bc+d^2
\end{bmatrix} \]
So,
\[tr(A^2)=a^2+2bc+d^2\Leftrightarrow \frac{1}{2}(tr(A^2))=\frac{a^2}{2}+bc+\frac{d^2}{2} \]
Finally,
\[\begin{vmatrix}
a & b\\
c & d
\end{vmatrix}=ad-bc \]
So,
\begin{align*}
\frac{1}{2}(tr(A))^2-\frac{1}{2}(tr(A^2))-det(A) &= (\frac{a^2}{2}+ad+\frac{d^2}{2})-(\frac{a^2}{2}+bc+\frac{d^2}{2})-(ad-bc)\\
&=(ad-bc)-(ad-bc)\\
&=0
\end{align*}
As desired.
\end{proof}
\item \textbf{Question 5.}
Let $L$ be the linear transformation from $\R^3\to \R^3$ given by,
\[L(x,y,z)=(-2x-y+z,-x-2y+z,x+y-2z)\]
We consider the construction of $L$ in matrix form, i.e.
\[L=\begin{bmatrix}
L_{11} & L_{12} & L_{13} \\
L_{21} & L_{22} & L_{23} \\
L_{31} & L_{32} & L_{33}
\end{bmatrix} \]
Such that,
\[\begin{bmatrix}
L_{11} & L_{12} & L_{13} \\
L_{21} & L_{22} & L_{23} \\
L_{31} & L_{32} & L_{33}
\end{bmatrix}\begin{pmatrix}
x\\
y\\
z
\end{pmatrix}=\begin{pmatrix}
-2x-y+z \\
-x-2y+z \\
x+y-2z
\end{pmatrix} \]
From this definition, it is clear that the rows of $L$ are constructed as the coefficients for each row of the output vector, thus,
\[L=\begin{bmatrix}
-2 & -1 & 1\\
-1 & -2 & 1\\
1 & 1 & -2
\end{bmatrix} \]
Consider now, the computations of the eigenvalues of this matrix as $det(L-\lambda I)=0$ So,
\begin{align*}
\det(L-\lambda I) &= \begin{vmatrix}
-2-\lambda & -1 & 1\\
-1 & -2-\lambda & 1\\
1 & 1 & -2-\lambda
\end{vmatrix}\\
&=(-2-\lambda)((-2-\lambda)^2-1)-(-1)(2+\lambda-1)+(1)(-1+2+\lambda)\\
&=-\lambda^3-6\lambda^2-9\lambda-4
\end{align*}
Plotting this characteristic polynomial, we see that $\lambda=-4$ is a root, as well as $\lambda=-1$ being a multiple root. Thus, we have our three roots of this polynomial, $\lambda_1=-4,\ \lambda_2=\lambda_3=-1$. We begin construction of the eigenvectors as follows,
\[(L-\lambda_1I)\vec{e}_1=\vec{0},\ (L-\lambda_2I)\vec{e}_2=\vec{0},\ (L-\lambda_3I)\vec{e}_3=\vec{0}\]
\begin{align*}
(L-\lambda_1I)\vec{e}_1 &=\begin{bmatrix}
2 & -1 & 1\\
-1 & 2 & 1\\
1 & 1 & 2
\end{bmatrix}\begin{bmatrix}
\varepsilon_1\\
\varepsilon_2\\
\varepsilon_3
\end{bmatrix}\\
&=\begin{bmatrix}
2 & -1 & 1\\
0 & \frac{3}{2} & \frac{3}{2}\\
0 & \frac{3}{2} & \frac{3}{2}
\end{bmatrix}\begin{bmatrix}
\varepsilon_1\\
\varepsilon_2\\
\varepsilon_3
\end{bmatrix} && R_2+=\frac{1}{2}R_1,\ R_3-=\frac{1}{2}R_1\\
&=\begin{bmatrix}
2 & -1 & 1\\
0 & 1 & 1\\
0 & 0 & 0
\end{bmatrix}\begin{bmatrix}
\varepsilon_1\\
\varepsilon_2\\
\varepsilon_3
\end{bmatrix} && R_2=\frac{2}{3}R_2,\ R_3-=R_2\\
&=\begin{bmatrix}
1 & 0 & 1\\
0 & 1 & 1\\
0 & 0 & 0
\end{bmatrix}\begin{bmatrix}
\varepsilon_1\\
\varepsilon_2\\
\varepsilon_3
\end{bmatrix} && R_1+=R_2,\ R_1=\frac{1}{2}R_1\\
\end{align*}
Setting equal to $\vec{0}$ we find,
\[\begin{bmatrix}
1 & 0 & 1\\
0 & 1 & 1\\
0 & 0 & 0
\end{bmatrix}\begin{bmatrix}
\varepsilon_1\\
\varepsilon_2\\
\varepsilon_3
\end{bmatrix}=\begin{bmatrix}
0\\
0\\
0
\end{bmatrix}\]
\[\varepsilon_1=-\varepsilon_3,\ \ \varepsilon_2=-\varepsilon_3\]
So, taking $\varepsilon_3=1$ for simplicity,
\[\lambda_1 \Rightarrow \vec{e}_1=\begin{bmatrix}
-1 \\
-1\\
1
\end{bmatrix}\]
Similarly,
\begin{align*}
(L-\lambda_{2,3}I)\vec{e}_{2,3}&=\begin{bmatrix}
-1 & -1 & 1\\
-1 & -1 & 1\\
1 & 1 & -1
\end{bmatrix}\begin{bmatrix}
\varepsilon_1\\
\varepsilon_2\\
\varepsilon_3
\end{bmatrix}\\
&=\begin{bmatrix}
1 & 1 & -1 \\
0 & 0 & 0 \\
0 & 0 & 0
\end{bmatrix} && R_2-=R_1,\ R_3+=R_1,\ R_1=-R_1
\end{align*}
Setting equal to $\vec{0}$ we find,
\[\begin{bmatrix}
1 & 1 & -1\\
0 & 0 & 0\\
0 & 0 & 0
\end{bmatrix}\begin{bmatrix}
\varepsilon_1\\
\varepsilon_2\\
\varepsilon_3
\end{bmatrix}=\begin{bmatrix}
0\\
0\\
0
\end{bmatrix}\]
\[\varepsilon_1+\varepsilon_2=\varepsilon_3\]
Here, we must fix two variables to arrive at a vector, thus we have multiple eigenvectors:
\[\lambda_2,\ \varepsilon_1=0,\ \varepsilon_2=1 \Rightarrow \vec{e}_2=\begin{bmatrix}
0\\
1\\
1
\end{bmatrix} \]
\[\lambda_3,\ \varepsilon_1=1,\ \varepsilon_2=0 \Rightarrow \vec{e}_3=\begin{bmatrix}
1\\
0\\
1
\end{bmatrix} \]
Finally, as a check, we apply the linear transformation $L$ onto each of these vectors to see if they are true eigenvectors.
\[L\vec{e}_1=\begin{bmatrix}
-2 & -1 & 1\\
-1 & -2 & 1\\
1 & 1 & -2
\end{bmatrix}\begin{bmatrix}
-1\\
-1\\
1
\end{bmatrix}=\begin{bmatrix}
2+1+1\\
1+2+1\\
-1-1-2
\end{bmatrix}=\begin{bmatrix}
4\\
4\\
-4
\end{bmatrix}=-4\begin{bmatrix}
-1\\
-1\\
1
\end{bmatrix}=\lambda_1\vec{e}_1 \]
\[L\vec{e}_2=\begin{bmatrix}
-2 & -1 & 1\\
-1 & -2 & 1\\
1 & 1 & -2
\end{bmatrix}\begin{bmatrix}
0\\
1\\
1
\end{bmatrix}=\begin{bmatrix}
0-1+1\\
0-2+1\\
0+1-2
\end{bmatrix}=\begin{bmatrix}
0\\
-1\\
-1
\end{bmatrix}=-1\begin{bmatrix}
0\\
1\\
1
\end{bmatrix}=\lambda_2\vec{e}_2 \]
\[L\vec{e}_3=\begin{bmatrix}
-2 & -1 & 1\\
-1 & -2 & 1\\
1 & 1 & -2
\end{bmatrix}\begin{bmatrix}
1\\
0\\
1
\end{bmatrix}=\begin{bmatrix}
-2+0+1\\
-1+0+1\\
1+0-2
\end{bmatrix}=\begin{bmatrix}
-1\\
0\\
-1
\end{bmatrix}=-1\begin{bmatrix}
1\\
0\\
1
\end{bmatrix}=\lambda_3\vec{e}_3 \]
Hence, we hace found the eigenvalues and the eigenvectors associated with them. To test whether these vectors are linearly independent, we need,
\[det([\vec{e}_1,\vec{e}_2,\vec{e}_3])\neq 0\]
Checking,
\[\begin{vmatrix}
-1 & 0 & 1\\
-1 & 1 & 0\\
1 & 1 & 1
\end{vmatrix}=-1(1-0)+0+1(-1-1)=-1-2=-3\neq 0\]
Thus, our eigenvectors are independent.
\end{description}
\end{document}

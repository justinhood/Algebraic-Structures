\documentclass[letterpaper,10pt]{article}
\usepackage[top=2cm, bottom=1.5cm, left=1cm, right=1cm]{geometry}
\usepackage{amsmath, amssymb, amsthm,graphicx}
\usepackage{fancyhdr,enumitem}
\pagestyle{fancy}

\lhead{\today}
\chead{Algebraic Structures HW 3}
\rhead{Justin Hood}

\newcommand{\Z}{\mathbb{Z}}
\newcommand{\Q}{\mathbb{Q}}
\newcommand{\R}{\mathbb{R}}
\newcommand{\C}{\mathbb{C}}
\newtheorem{lem}{Lemma}

\begin{document}
\begin{description}
\item[1] Compute the products\\
\begin{enumerate}[label=\alph*.]
\item
\[\begin{pmatrix}
1 & 2 & 1\\
4 & 5 & 2\\
7 & 8 & 2
\end{pmatrix}\begin{pmatrix}
-2 & \frac{4}{3} & \frac{-1}{3}\\
2 & \frac{-5}{3} & \frac{2}{3}\\
-1 & 2 &-1
\end{pmatrix}=\begin{pmatrix}
1 & 0 & 0\\
0 & 1 & 0\\
0 & 0 & 1
\end{pmatrix}
\]
\item
\[\begin{pmatrix}
1 & 2 & 3 & 4 & 5
\end{pmatrix}
\begin{pmatrix}
1 \\
2 \\
3 \\
4 \\
5
\end{pmatrix}=55
\]
\item
\[\begin{pmatrix}
1 \\
2 \\
3 \\
4 \\
5
\end{pmatrix}
\begin{pmatrix}
1 & 2 & 3 & 4 & 5
\end{pmatrix}=\begin{pmatrix}
1 & 2 & 3 & 4 & 5\\
2 & 4 & 6 & 8 & 10\\
3 & 6 & 9 & 12 & 15\\
4 & 8 & 12 & 16 & 20\\
5 & 10 & 15 & 20 & 25\\
\end{pmatrix}
\]
\item
\[\begin{pmatrix}
1 & 2 & 1\\
4 & 5 & 2\\
7 & 8 & 2
\end{pmatrix}\begin{pmatrix}
-2 & \frac{4}{3} & \frac{-1}{3}\\
2 & \frac{-5}{3} & \frac{2}{3}\\
-1 & 2 &-1
\end{pmatrix}\begin{pmatrix}
1 & 2 & 1\\
4 & 5 & 2\\
7 & 8 & 2
\end{pmatrix}=\begin{pmatrix}
1 & 0 & 0\\
0 & 1 & 0\\
0 & 0 & 1
\end{pmatrix}\begin{pmatrix}
1 & 2 & 1\\
4 & 5 & 2\\
7 & 8 & 2
\end{pmatrix}=\begin{pmatrix}
1 & 2 & 1\\
4 & 5 & 2\\
7 & 8 & 2
\end{pmatrix}
\]
\item
\[\begin{pmatrix}
x & y & z
\end{pmatrix}\begin{pmatrix}
2 & 1 & 1\\
1 & 2 & 1\\
1 & 1 & 2
\end{pmatrix}\begin{pmatrix}
x\\
y\\
z
\end{pmatrix}=2(x^2+y^2+z^2+xy+yz+xz)
\]
\item
\[\begin{pmatrix}
2 & 1 & 2 & 1 & 2\\
0 & 2 & 1 & 2 & 1\\
0 & 1 & 2 & 1 & 2\\
0 & 2 & 1 & 2 & 1\\
0 & 0 & 0 & 0 & 2
\end{pmatrix}\begin{pmatrix}
1 & 2 & 1 & 2 & 1\\
0 & 1 & 2 & 1 & 2\\
0 & 2 & 1 & 2 & 1\\
0 & 1 & 2 & 1 & 2\\
0 & 0 & 0 & 0 & 1
\end{pmatrix}=\begin{pmatrix}
2 & 10 & 8 & 10 & 10\\
0 & 6 & 9 & 6 & 10\\
0 & 6 & 6 & 6 & 8\\
0 & 6 & 9 & 6 & 10\\
0 & 0 & 0 & 0 & 2
\end{pmatrix}
\]
\item
\[\begin{pmatrix}
-2 & \frac{4}{3} & \frac{-1}{3}\\
2 & \frac{-5}{3} & \frac{2}{3}\\
-1 & 2 &-1
\end{pmatrix}\begin{pmatrix}
4 & \frac{2}{3} & \frac{-2}{3}\\
6 & \frac{5}{3} & \frac{-2}{3}\\
12 & \frac{-16}{3} & \frac{10}{3}
\end{pmatrix}\begin{pmatrix}
1 & 2 & 1\\
4 & 5 & 2\\
7 & 8 & 2
\end{pmatrix}=\begin{pmatrix}
-4 & \frac{8}{3} & \frac{-2}{3}\\
6 & -5 & 2\\
4 & \frac{28}{3} & \frac{-16}{3}
\end{pmatrix}\begin{pmatrix}
1 & 2 & 1\\
4 & 5 & 2\\
7 & 8 & 2
\end{pmatrix}=\begin{pmatrix}
2 & 0 & 0\\
0 & 3 & 0\\
4 & 12 & 12
\end{pmatrix}
\]
\end{enumerate}
\item[3] Symmetry
\begin{enumerate}[label=\alph*.]
\item Let,
\[A=\begin{pmatrix}
1 & 2 & 0\\
3 & -1 & 4
\end{pmatrix}\]
We now compute, $AA^T$ and $A^TA$.
\begin{align*}
AA^T &= \begin{pmatrix}
1 & 2 & 0\\
3 & -1 & 4
\end{pmatrix}\begin{pmatrix}
1 & 3\\
2 & -1\\
0 & 4
\end{pmatrix}=\begin{pmatrix}
5 & 1\\
1 & 26
\end{pmatrix}\\
A^TA &= \begin{pmatrix}
1 & 3\\
2 & -1\\
0 & 4
\end{pmatrix}=\begin{pmatrix}
1 & 2 & 0\\
3 & -1 & 4
\end{pmatrix}\begin{pmatrix}
10 & -1 & 12\\
-1 & 5 & -4\\
12 & -4 & 16
\end{pmatrix}
\end{align*}
\item Let $M$ be any $m\times n$ matrix. Show that $M^TM$ and $MM^T$ are symmetric.\\
\begin{proof}
Consider, by definition of symmetry, a matrix $A$ is symmetric if and only if $A=A^T$. Then,
\begin{align*}
(M^TM)^T &= M^T(M^T)^T && \text{By definition of transpose}\\
\Leftrightarrow &= M^TM && \text{Transpose of a transpose is the original matrix}
\end{align*}
And,
\begin{align*}
(MM^T)^T &= (M^T)^TM^T && \text{By definition of transpose}\\
\Leftrightarrow &= MM^T && \text{Transpose of a transpose is the original matrix}
\end{align*}
So,
\[(M^TM)^T=M^TM,\text{ and},\ (MM^T)^T=MM^T\]
Thus, these products are symmetric.
\end{proof}
Next, we consider the size of the resultant matrices,
$M^TM=(n\times m)\times(m\times n)=n\times n$ and $MM^T=(m\times n)\times (n\times m)=m\times m$.\\\\
Finally, we consider the traces of both $M^TM$ and $MM^T$. We note that in the text that we proved that,
\[tr(AB)=tr(BA)\Rightarrow tr(M^TM)=tr(MM^T)\]
thus, their traces are equal.
\end{enumerate}
\item[4] Dot product\\
Consider, $x=\begin{pmatrix}
x_1 \\
\vdots\\
x_n
\end{pmatrix}$ and $y=\begin{pmatrix}
y_1\\
\vdots\\
y_n
\end{pmatrix}$, column vectors. Prove that $x\cdot y=x^TIy$
\begin{proof}
Consider first, $x\cdot y$.
\[x\cdot y=\sum_{i=1}^nx_iy_i\]
Now, we consider, $x^TI$. By definition of $I$, this is $x^T$. Then,
\[x^TIy=x^Ty=\sum_{i=1}^nx_iy_i\]
Thus, we see that $x\cdot y=x^TIy$
\end{proof}
\item[6]
\end{description}
\end{document}

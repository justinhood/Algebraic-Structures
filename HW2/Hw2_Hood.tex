\documentclass[letterpaper,10pt]{article}
\usepackage[top=2cm, bottom=1.5cm, left=1cm, right=1cm]{geometry}
\usepackage{amsmath, amssymb, amsthm}
\usepackage{fancyhdr, enumitem}
\pagestyle{fancy}

\lhead{\today}
\chead{Algebraic Structures}
\rhead{Justin Hood}

\newcommand{\Z}{\mathbb{Z}}
\newcommand{\Q}{\mathbb{Q}}
\newcommand{\C}{\mathbb{C}}
\newcommand{\R}{\mathbb{R}}
\newcommand{\vect}[1]{\boldsymbol{#1}}
\newcommand{\twoby}[2]{\begin{pmatrix}
#1 \\
#2
\end{pmatrix}}
\newcommand{\conj}{\bar{z}}
\begin{document}
\begin{description}
\item[6.3]\hfill\\
\begin{enumerate}[label=\alph*.]
\item If $p\twoby{1}{2}=1$ and $p\twoby{2}{4}=3$ is $p$ linear?\\
If $p$ is a linear function, then
\[p(cx)=cp(x)\]
Assuming to the contrary that $p$ is linear, we note,
\[p\twoby{2}{4}=2p\twoby{1}{2}\]
But,
\[2p\twoby{1}{2}=2(1)=2\neq 3\]
Thus, linearity has been violated. So $p$ is not linear.
\item If $Q(x^2)=x^3$ and $Q(2x^2)=x^4$ is $Q$ linear?\\
If $Q$ is a linear function, then
\[Q(cx^2)=cQ(x^2)\]
Assuming to the contrary that $Q$ is linear, we note,
\[Q(2x^2)=2Q(x^2)\]
But,
\[2Q(x^2)=2x^3\neq x^4,\text{ in general}\]
Since this does not hold for $x\neq 2$, $Q$ is not linear in general.
\end{enumerate}
\item[6.5]\hfill\\
Let $P_n$ be the space of polynomials of degree $n$ or less on $t$. Let $L:P_2\to P_3$ such that $L(1)=4,\ L(t)=t^3,\ L(t^2)=t-1$.
\begin{enumerate}[label=\alph*.]
\item \begin{align*}
L(1+t+2t^2) &=L(1)+L(t)+L(2t^2)\\
&=L(1)+L(t)+2L(t^2)\\
&=4+t^3+2(t-1)\\
&=t^3+2t+2
\end{align*}
\item \begin{align*}
L(a+bt+ct^2) &=L(a)+L(bt)+L(ct^2)\\
&=aL(1)+bL(t)+cL(t^2)\\
&=a(4)+bt^3+c(t-1)\\
&=bt^3+ct+4a-c
\end{align*}
\item Find $a,b,c$ such that, $L(a+bt+ct^2)=1+3t+2t^3$\\\\
From above, we have $L(a+bt+ct^2)=bt^3+ct+4a-c$. Then,
\begin{align*}
bt^3+ct+4a-c&=1+3t+2t^3
\end{align*}
So, we have
\begin{align*}
b&=2\\
c&=3\\
4a-c&=1\\
\Leftrightarrow 4a-3&=1\\
\Leftrightarrow a&=1
\end{align*}
\end{enumerate}
\item[6.6]\hfill\\
Let $\mathcal{I}:f\to\mathcal{I}f(x)$ where, $\mathcal{I}f(x):=\int_{0}^{x}f(t)dt$, where $f$ is a continuous function. Then we shall consider,
\begin{align*}
\mathcal{I}(af+bg) &= \int_{0}^{x}af(t)+bg(t)dt\\
&=\int_{0}^{x}af(t)dt+\int_{0}^{x}bg(t)dt\\
&=a\int_{0}^{x}f(t)dt+b\int_{0}^{x}g(t)dt\\
&=a\mathcal{I}f+b\mathcal{I}g
\end{align*}
As required by the definition of a linear operator. Thus, $\mathcal{I}$ is indeed a linear operator on the space of continuous functions.
\item[6.7]\hfill\\
Let $z\in\C$, and let $\conj=x-iy$ and $c:\R^2\to\R^2$ such that $c(x,y)=(x,-y)$.
\begin{enumerate}[label=\alph*.]
\item Consider the following, $\alpha=(ax,ay)$ and $\beta=(bx,by)$. Then,
\begin{align*}
c(\alpha+\beta)&=c(ax+bx,ay+by)\\
&=(ax+bx,-(ay+by))\\
&=(ax,-ay)+(bx,-by)\\
&=ac(x,y)+bc(x,y)
\end{align*}
As required by the definition of a linear map. Thus, $c$ is a linear map from $\R^2\to\R^2$.
\item Consider the conjugate operator $\conj$. Let us assume that $\conj$ is a linear operator on $\C$. Then,
\[\conj(\alpha+\beta)=\conj(\alpha)+\conj(\beta)\]
\[\conj(c\alpha)=c\conj(\alpha),\ c\in\C\]
Consider now, $\alpha=x+iy,\ c=i,\ i^2=-1$. Then,
\[\conj(i\alpha)=\conj(ix-y)=-y-ix\]
But,
\[i\conj(\alpha)=i\big(x-iy\big)=y+ix\]
A clear contradiction to the definition of a linear operator. Thus, $\conj$ cannot be a linear operator on $\C$.
\end{enumerate}
\end{description}
\end{document}

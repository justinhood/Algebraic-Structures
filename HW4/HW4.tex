\documentclass[letterpaper,10pt]{article}
\usepackage[top=2cm, bottom=1.5cm, left=1cm, right=1cm]{geometry}
\usepackage{amsmath, amssymb, amsthm,graphicx,tikz}
\usetikzlibrary{arrows}
\usepackage{fancyhdr}
\pagestyle{fancy}

\lhead{\today}
\chead{MATH 720 Homework}
\rhead{Justin Hood}

\newcommand{\Z}{\mathbb{Z}}
\newcommand{\Q}{\mathbb{Q}}
\newcommand{\R}{\mathbb{R}}
\newcommand{\C}{\mathbb{C}}
\newtheorem{lem}{Lemma}

\begin{document}
\begin{enumerate}
\item Determine if $x-x^3\in span\{x^2,2x+x^2,x+x^3\}$\\
Let $\alpha =x^2$, $\beta =2x+x^2$, and $\gamma =x+x^3$, and consider,
\[-\alpha+\beta-\gamma=-(x^2)+(2x+x^2)-(x+x^3)=(2x-x)+(x^2-x^2)+(-x^3)=x-x^3\]
Thus, we see that $x-x^3$ is the linear combination of elements of the set, and thus is in the span.
\item Let $U$ and $W$ be subspaces of $V$. Are:
\begin{enumerate}
\item $U\cup W$
\item $U\cap W$
\end{enumerate}

\begin{enumerate}
\item No. Consider the vector space, $\R^3$, with subspaces consisting of the x-axis and y-axis respectibely. i.e. $U=\{\{x,0,0\}|x\in \R\}$, $W=\{\{0,y,0\}|y\in \R\}$ (Shown below)
\begin{center}
\resizebox{3in}{3in}{%
\begin{tikzpicture}[x=0.5cm,y=0.5cm,z=0.3cm,>=stealth]
% The axes
\draw[very thick, blue] (xyz cs:x=-13.5) -- (xyz cs:x=13.5) node[above] {$x$};
\draw[->] (xyz cs:y=-13.5) -- (xyz cs:y=13.5) node[right] {$y$};
\draw[->] (xyz cs:z=-13.5) -- (xyz cs:z=13.5) node[above] {$z$};
% The thin ticks
\foreach \coo in {-13,-12,...,13}
{
  \draw (\coo,-1.5pt) -- (\coo,1.5pt);
  \draw (-1.5pt,\coo) -- (1.5pt,\coo);
  \draw (xyz cs:y=-0.15pt,z=\coo) -- (xyz cs:y=0.15pt,z=\coo);
}
% The thick ticks
% Dashed lines for the points P, Q

% Dots and labels for P, Q
\node[fill,circle,inner sep=1.5pt,label={}] at (0,0,0) {};
%\node[fill,circle,inner sep=1.5pt,label={above:$P(3,0,5)$}] at (3,5) {};
% The origin
\end{tikzpicture}
}\resizebox{3in}{3in}{%
\begin{tikzpicture}[x=0.5cm,y=0.5cm,z=0.3cm,>=stealth]
% The axes
\draw[->] (xyz cs:x=-13.5) -- (xyz cs:x=13.5) node[above] {$x$};
\draw[very thick, blue] (xyz cs:y=-13.5) -- (xyz cs:y=13.5) node[right] {$y$};
\draw[->] (xyz cs:z=-13.5) -- (xyz cs:z=13.5) node[above] {$z$};
% The thin ticks
\foreach \coo in {-13,-12,...,13}
{
  \draw (\coo,-1.5pt) -- (\coo,1.5pt);
  \draw (-1.5pt,\coo) -- (1.5pt,\coo);
  \draw (xyz cs:y=-0.15pt,z=\coo) -- (xyz cs:y=0.15pt,z=\coo);
}
% The thick ticks
% Dashed lines for the points P, Q

% Dots and labels for P, Q
\node[fill,circle,inner sep=1.5pt,label={}] at (0,0,0) {};
%\node[fill,circle,inner sep=1.5pt,label={above:$P(3,0,5)$}] at (3,5) {};
% The origin
\end{tikzpicture}}
\end{center}
Then, we consider their union (shown below)
\begin{center}
\resizebox{3in}{3in}{%
\begin{tikzpicture}[x=0.5cm,y=0.5cm,z=0.3cm,>=stealth]
% The axes
\draw[very thick, blue] (xyz cs:x=-13.5) -- (xyz cs:x=13.5) node[above] {$x$};
\draw[very thick, blue] (xyz cs:y=-13.5) -- (xyz cs:y=13.5) node[right] {$y$};
\draw[->] (xyz cs:z=-13.5) -- (xyz cs:z=13.5) node[above] {$z$};
% The thin ticks
\foreach \coo in {-13,-12,...,13}
{
  \draw (\coo,-1.5pt) -- (\coo,1.5pt);
  \draw (-1.5pt,\coo) -- (1.5pt,\coo);
  \draw (xyz cs:y=-0.15pt,z=\coo) -- (xyz cs:y=0.15pt,z=\coo);
}
% The thick ticks
% Dashed lines for the points P, Q

% Dots and labels for P, Q
\node[fill,circle,inner sep=1.5pt,label={}] at (0,0,0) {};
%\node[fill,circle,inner sep=1.5pt,label={above:$P(3,0,5)$}] at (3,5) {};
% The origin
\end{tikzpicture}}
\end{center}
Consider now the elements $\{7,0,0\}\in U$ and $\{0,5,0\}\in W$
\begin{center}
\resizebox{3in}{3in}{%
\begin{tikzpicture}[x=0.5cm,y=0.5cm,z=0.3cm,>=stealth]
% The axes
\draw[very thick, blue] (xyz cs:x=-13.5) -- (xyz cs:x=13.5) node[above] {$x$};
\draw[very thick, blue] (xyz cs:y=-13.5) -- (xyz cs:y=13.5) node[right] {$y$};
\draw[->] (xyz cs:z=-13.5) -- (xyz cs:z=13.5) node[above] {$z$};
% The thin ticks
\foreach \coo in {-13,-12,...,13}
{
  \draw (\coo,-1.5pt) -- (\coo,1.5pt);
  \draw (-1.5pt,\coo) -- (1.5pt,\coo);
  \draw (xyz cs:y=-0.15pt,z=\coo) -- (xyz cs:y=0.15pt,z=\coo);
}
% The thick ticks
% Dashed lines for the points P, Q

% Dots and labels for P, Q
\node[fill,circle,inner sep=1.5pt,label={}] at (0,0,0) {};
\node[fill,circle,color=red,inner sep=4pt,label={(7,0,0)}] at (7,0,0) {};
\node[fill,circle,color=red,inner sep=4pt,label={(0,5,0)}] at (0,5,0) {};
% The origin
\end{tikzpicture}}
\end{center}
Which are both contained in the union. But their sum, $\{7,5,0\}$ is not contained in the union as seen below.
\begin{center}
\resizebox{3in}{3in}{%
\begin{tikzpicture}[x=0.5cm,y=0.5cm,z=0.3cm,>=stealth]
% The axes
\draw[very thick, blue] (xyz cs:x=-13.5) -- (xyz cs:x=13.5) node[above] {$x$};
\draw[very thick, blue] (xyz cs:y=-13.5) -- (xyz cs:y=13.5) node[right] {$y$};
\draw[->] (xyz cs:z=-13.5) -- (xyz cs:z=13.5) node[above] {$z$};
% The thin ticks
\foreach \coo in {-13,-12,...,13}
{
  \draw (\coo,-1.5pt) -- (\coo,1.5pt);
  \draw (-1.5pt,\coo) -- (1.5pt,\coo);
  \draw (xyz cs:y=-0.15pt,z=\coo) -- (xyz cs:y=0.15pt,z=\coo);
}
% The thick ticks
% Dashed lines for the points P, Q

% Dots and labels for P, Q
\node[fill,circle,inner sep=1.5pt,label={}] at (0,0,0) {};
\node[fill,circle,color=red,inner sep=4pt,label={(7,5,0)}] at (7,5,0) {};
% The origin
\end{tikzpicture}}
\end{center}
This violates the closure under addition property of a vector space. Thus, we conclude the the union of two vector spaces is not a subspace in general.
\item Let $\eta=U\cup W$, where $U,W\subseteq V$. Let us now consider whether $\eta$ is itself a subspace of $V$. First, we note, $\vec{0}\in U,W$. Then, by construction, $\vec{0}\in\eta$. Then, we know the zero element is in $\eta$. Next, consider an arbitrary $e\in \eta$. Then, we know that $e\in U \wedge e\in W$. Because $U$ and $W$ are both subspaces, $\forall e \in U,W\ \lambda e\in U,W$ for a scalar $\lambda$. Because $\lambda e\in U,W\ \lambda e \in \eta$. Thus, $\eta$ is closed under scalar multiplication. Finally, consider $x,y\in \eta$. Then, $x,y\in U,W\Rightarrow x+y\in U,W$ because they are both subspaces. Thus, $x+y\in \eta$. Thus it is closed under addition. So, we see that $\eta$ is a subset as well.
\end{enumerate}\
\item Let $L:\R^3\to \R^3$ where,
\[L(x,y,z)=(x+2y+z,\ 2x+y+z,\ 0)\]
Find $ker(L)$, $im(L)$ and the eigenspaces of $\R^3_{-1},\R^3_3$.
\begin{enumerate}
\item To begin, we construct the matrix defined by $L$ as,
\[\begin{bmatrix}
1 & 2 & 1 \\
2 & 1 & 1 \\
0 & 0 & 0
\end{bmatrix} \]
We then set this matrix equal to the zero vector and reduce,
\[\left[ \begin{array}{ccc|c}
1 & 2 & 1 & 0\\
2 & 1 & 1 & 0\\
0 & 0 & 0 & 0
\end{array}\right] \]
\[\left[ \begin{array}{ccc|c}
1 & 2 & 1 & 0\\
0 & -3 & -1 & 0\\
0 & 0 & 0 & 0
\end{array}\right]\]
\[\left[ \begin{array}{ccc|c}
1 & 0 & \frac{1}{3} & 0\\
0 & 1 & \frac{1}{3} & 0\\
0 & 0 & 0 & 0
\end{array}\right] \]
From this reduction, we glean the following equations,
\[v_1=-\frac{1}{3}v_3,\ v_2=-\frac{1}{3}v_3\]
Thus, the kernel of the transformation is,
\[\lambda \begin{bmatrix}
-1\\
-1\\
3
\end{bmatrix},\ \lambda\in \R\]
\item We now consider the image of $L$ as,
\[im(L)=\begin{bmatrix}
1 & 2 & 1\\
2 & 1 & 1\\
0 & 0 & 0
\end{bmatrix} \begin{bmatrix}
x\\
y\\
z
\end{bmatrix}=\begin{bmatrix}
x+2y+z\\
2x+y+z\\
0
\end{bmatrix}\]
\item Finally, we consider the two eigenspaces.
\begin{enumerate}
\item $\R^3_{-1}:$ Consider the difference, $(L+I)v=0$
\[L+I=\begin{bmatrix}
2 & 2 & 1\\
2 & 2 & 1\\
0 & 0 & 1
\end{bmatrix} \]
Then,
\[\left[ \begin{array}{ccc|c}
2 & 2 & 1 & 0\\
0 & 0 & 0 & 0\\
0 & 0 & 1 & 0
\end{array}\right] \]
\[\left[ \begin{array}{ccc|c}
1 & 1 & 0 & 0\\
0 & 0 & 1 & 0\\
0 & 0 & 0 & 0
\end{array}\right] \]
Then we arrive at the equations,
\[v_1=-v_2,\ v_3=0\]
So our eigenvector is,
\[\eta \begin{pmatrix}
-1\\
1\\
0
\end{pmatrix}, \eta\in \R \]
So the eigenspace is $span\{\eta \begin{pmatrix}
-1\\
1\\
0
\end{pmatrix}| \eta\in \R\}$.
\item $R_3^3:$ Consider the difference, $(L-3I)v=0$
\[L-3I=\begin{bmatrix}
-2 & 2 & 1\\
2 & -2 & 1\\
0 & 0 & 3
\end{bmatrix} \]
Then,
\[\left[ \begin{array}{ccc|c}
-2 & 2 & 1 & 0\\
2 & -2 & 1 & 0\\
0 & 0 & 3 & 0
\end{array}\right] \]
\[\left[ \begin{array}{ccc|c}
1 & -1 & 0 & 0\\
0 & 0 & 1 & 0\\
0 & 0 & 0 & 0
\end{array}\right] \]
Then we arrive at the equations,
\[v_1=v_2,\ v_3=0\]
So our eigenvector is,
\[\eta \begin{pmatrix}
1\\
1\\
0
\end{pmatrix}, \eta\in \R \]
So the eigenspace is $span\{\eta \begin{pmatrix}
1\\
1\\
0
\end{pmatrix}| \eta\in \R\}$.
\end{enumerate}
\end{enumerate}
\end{enumerate}
\end{document}
